% !TEX program = xelatex
\documentclass[24pt]{article}
\usepackage{xeCJK}
\usepackage{amsmath}
\usepackage{amssymb}
\usepackage{indentfirst}
% \usepackage{geometry}
% \geometry{a4paper,scale=0.75}
\author{2017211123 褚逸豪}
\title{n阶汉诺塔移动次数}
\begin{document}
    \maketitle
    \section{符号}
    我们记$T_n$为根据规则将n个圆盘从一根柱子移动到另一根指定的柱子所需要的最少移动次数。记三根柱子分别为A,B,C,我们要将所有的圆盘从A移动到C。
    \section{推导递推关系}
    我们通过分解问题,n阶汉诺塔问题可以降阶为n-1阶汉诺塔问题。因为我们实现任务的最优方法是先将最小的n-1个盘片移动到B,再将最大的盘片移动到C,最后将B上的盘片移动到C。我们可以不停的降阶,直到问题简化为到边界——0阶汉诺塔问题,你什么都不用做。\par
    根据以上关系,我们轻松的得到${T_n}$的递推关系及初始条件
    $$\begin{array}{l}
        T_0=0;\\
        T_n=2T_{n-1}+1,\ n>0.
    \end{array}$$
    \section{do some math tricks}
    通过高中数学的数列处理技巧,我们对式子进行如下变形
    $$\begin{array}{l}\begin{aligned}
        T_n&=2T_{n-1}+1;\\
        T_n+1&=2T_{n-1}+2;\\
        T_n+1&=2(T_{n-1}+1);\\
        T_0+1&=1;\\
        T_n+1&=2^{n};\\
        T_n&=2^n-1.
    \end{aligned}\end{array}$$
    \par
    至此,我们已经成功的推导出了n阶汉诺塔所需要最少的移动次数——$2^n-1$。
\end{document}