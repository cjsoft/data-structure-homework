% ! TEX PROGRAM = xelatex
\documentclass[UTF8]{ctexart}
    \usepackage{amsmath}
    \usepackage{amssymb}
    \usepackage{geometry}
    \usepackage{multicol}
    \geometry{a4paper,scale=0.7}
    \title{广义表运算}
    \author{2017211123 褚逸豪}
\begin{document}
    \maketitle
    \begin{multicols}{2}\noindent
        \section{约定}
        多元素元组表示方式:$(x_1,x_2,x_3,\ldots)$\par
        单元素元组表示方式:$(x,)$
        \section{求广义表运算的结果}
        $$\begin{aligned}
            Head[((a,b),(c,d))]&=(a,b)\\
            Tail[((a,b),(c,d))]&=((c,d),)\\
            Head[Tail[((a,b),(c,d))]]&=(c,d)\\
            Tail[Head[((a,b),(c,d))]]&=(b,)\\
            Head[Tail[Head[((a,b),(c,d))]]]&=b\\
            Tail[Head[Tail[((a,b),(c,d))]]]&=(d,)
        \end{aligned}$$
        \section{利用Head和Tail运算取出原子c}
        $$\begin{aligned}
            L_1&=(a,b,c,d);\\
            c&=Head[Tail[Tail[L_1]]].\\
            L_2&=((a,b),(c,d));\\
            c&=Head[Head[Tail[L_2]]].\\
            L_3&=(((a,),(b,),(c,),(d,)));\\
            c&=Head[Head[Tail[Tail[L_3]]]].\\
            L_4&=(a,(b,),((c,),),(((d,),),));\\
            c&=Head[Head[Head[Tail[Tail[L_4]]]]].\\
            L_5&=((((a,),),),((b,),),(c,),d);\\
            c&=Head[Head[Tail[Tail[L_5]]]].\\
            L_6&=((((a,),b),c),d);\\
            c&=Head[Tail[Head[L_6]]].\\
            L_7&=(a,(b,(c,(d,))));\\
            c&=Head[Head[Tail[Head[Tail[L_7]]]]].\\
            L_8&=(a,(b,(c,),d));\\
            c&=Head[Head[Tail[Head[Tail[L_8]]]]].
        \end{aligned}$$    
    \end{multicols}
    
\end{document}