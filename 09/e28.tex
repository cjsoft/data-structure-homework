% TEX PROGRAM = xelatex
\documentclass[UTF8]{ctexart}
    \title{哈希表相关问题}
    \author{2017211123 褚逸豪}
    \usepackage{amsmath}
    \usepackage{amssymb}
    \usepackage{geometry}
    \geometry{a4paper,scale=0.8}
\begin{document}
    \maketitle
    1到3节均基于假设:数据是按照原顺序插入散列表的。
    \section{计算H(x)}
    \begin{center}
        \begin{tabular}{|c|c|c|c|}
            \hline
            源关键字 & H(x) & 源关键字 & H(x)\\
            \hline
            Jan & 4 & Jul & 4\\
            \hline
            Feb & 2 & Aug & 0\\
            \hline
            Mar & 6 & Sep & 9\\
            \hline
            Apr & 0 & Oct & 7\\
            \hline
            May & 6 & Nov & 6\\
            \hline
            Jun & 4 & Dec & 1\\
            \hline
        \end{tabular}
    \end{center}
    \section{使用线性探测法处理冲突}
    \begin{center}
        \begin{tabular}{|c|c|c|c|c|c|c|c|c|c|c|c|c|c|c|c|c|}
            \hline
            0&1&2&3&4&5&6&7&8&9&10&11&12&13&14&15&16\\
            \hline
            Apr&Aug&Feb&Dec&Jan&Jun&Mar&May&Jul&Sep&Oct&Nov& & & & &\\
            \hline
        \end{tabular}
    \end{center}
    \section{使用链地址法处理冲突}
    \begin{center}
        \begin{tabular}{|c|llll}
            \cline{1-1}
            0&$\rightarrow$Apr&$\rightarrow$Aug&$\bigwedge$   \\
           \cline{1-1}
            1&$\rightarrow$Dec&$\bigwedge$  \\
           \cline{1-1}
             2&$\rightarrow$Feb&$\bigwedge$   \\
           \cline{1-1}
             3&$\bigwedge$   \\
           \cline{1-1}
             4&$\rightarrow$Jan&$\rightarrow$Jun&$\rightarrow$Jul$\bigwedge$   \\
           \cline{1-1}
             5&$\bigwedge$   \\
           \cline{1-1}
             6&$\rightarrow$Mar&$\rightarrow$May&$\rightarrow$Nov$\bigwedge$   \\
           \cline{1-1}
             7&$\rightarrow$Oct &$\bigwedge$   \\
           \cline{1-1}
             8&$\bigwedge$   \\
           \cline{1-1}
             9&$\rightarrow$Sep &$\bigwedge$   \\
           \cline{1-1}
             10&$\bigwedge$  \\
           \cline{1-1}
             11&$\bigwedge$ \\
           \cline{1-1}
             12&$\bigwedge$ \\
           \cline{1-1}
             13&$\bigwedge$ \\
           \cline{1-1}
             14&$\bigwedge$ \\
           \cline{1-1}
             15&$\bigwedge$ \\
           \cline{1-1}
             16&$\bigwedge$ \\
           \cline{1-1}
             
        \end{tabular}
    \end{center}
    \section{平均查找长度}
    \subsection{线性探测法}
    \subsubsection{查找成功}
    % $$ASL(12)=\frac{1}{12}(1\times 5+2\times 3+3\times 1+4\times 1+5\times 2)=\frac{29}{12}$$
    $$S_{nl}\approx \frac{1}{2} \left( 1+\frac{1}{1-\frac{12}{17}}\right)=\frac{11}{5}=2.2$$
    \subsubsection{查找失败}
    $$U_{nl}\approx \frac{1}{2} \left( 1+\frac{1}{\left(1-\frac{12}{17}\right)^2}\right)=\frac{157}{25}=6.28$$
    \subsection{链地址法}
    \subsubsection{查找成功}
    % $$ASL(12)=\frac{1}{12}(1\times 5+2\times 3+3\times 1+4\times 1+5\times 2)=\frac{29}{12}$$
    $$S_{ne}\approx 1+\frac{\frac{12}{17}}{2}=\frac{23}{17}\approx 1.35$$
    \subsubsection{查找失败}
    $$U_{ne}\approx \frac{12}{17}+e^{-\frac{12}{17}}\approx 3.42$$
\end{document}