\documentclass[UTF8]{ctexart}
    \usepackage{amsmath}
    \usepackage{amssymb}
    \usepackage{geometry}
    \title{前k大值}
    \author{2017211123 褚逸豪}

\begin{document}
    \maketitle
    \section{哪些方法可用}
    \begin{itemize}
        \item 取k次最大值并从序列中删除\\
        \item 最大堆xtract前k个元素\\
        \item 最多只能有k个元素的最小堆,当添加k+1个元素时,需要弹出堆顶元素\\
        \item 建平衡树\\
        \item 递归partion直到分割点相对于数组开始位置为k\\
        \item 冒泡k次
    \end{itemize}
    \section{方法分析}
    \subsection{k次最大值}
    显而易见,复杂度为O(kn)
    \subsection{最大堆方法}
    相当显然,复杂度也是O((k+n) log(n))
    \subsection{最小堆方法}
    复杂度为O(n log(k))
    \subsection{建平衡树}
    复杂度为O(n log(n))
    \subsection{递归partion}
    复杂度为O(n)
    \subsection{冒泡}
    复杂度为O(kn)
    \section{优秀的方法}
    当然是线性的递归partion做法啊
\end{document}
